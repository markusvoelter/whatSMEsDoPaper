% This is samplepaper.tex, a sample chapter demonstrating the
% LLNCS macro package for Springer Computer Science proceedings;
% Version 2.20 of 2017/10/04
%
\documentclass[runningheads]{llncs}
%
\usepackage{graphicx}

\usepackage{amsmath}
\usepackage{amssymb}
\usepackage{hyperref} 
\usepackage{soul}  
\usepackage{color}
\usepackage{xcolor}
\usepackage{graphicx}
\usepackage{inputenc} 
\usepackage{comment}
%\usepackage{wrapfig} 
\usepackage{ae}  
\usepackage{sidecap}
\usepackage{url}
\usepackage[T1]{fontenc} 
\usepackage{verbatim}
\usepackage{listings} 
\usepackage{tikz}
\usepackage{etoolbox}
\usepackage{mdframed}
\usepackage{tabularx}
\usepackage{multirow}
\usepackage{setspace}

\newenvironment{code}
    {\noindent
     \rule{\columnwidth}{0.6pt}
     \vspace{-4mm}
     \scriptsize \verbatim}
    {\endverbatim \normalsize
     \vspace{-4mm} 
    \rule{\columnwidth}{0.6pt} 
 	}
\usepackage{wrapfig}
\usepackage{relsize}
\usepackage{microtype}
\usepackage{multirow}
\usepackage{semantic}

\usepackage{booktabs}
\usepackage{paralist}
% important for nicer code font
\usepackage[scaled]{beramono}

\definecolor{lightlightgray}{gray}{0.9}
\definecolor{codecolor}{gray}{0.15} 
\definecolor{javadocblue}{rgb}{0,0,0.8} % javadoc
\definecolor{blackblack}{rgb}{0,0,0}
  
\definecolor{myGray1}{rgb}{0.85,0.85,.85}

\lstset{
basicstyle=\ttfamily\scriptsize\color{codecolor}, 
keywordstyle=\color{blackblack}\bfseries,       
% keywordstyle=\color{javadocblue},     
commentstyle=\color{gray},              
numbers=none,                           
numberstyle=\tiny,                      % Line-numbers fonts
stepnumber=1,                           % Step between two line-numbers
numbersep=5pt,                          % How far are line-numbers from code
%backgroundcolor=\color{lightlightgray}, % Choose background color
tabsize=2,     
frame=single,
%xleftmargin=1.5mm,
%xrightmargin=1.5mm,
captionpos=b,   
keepspaces=true,
xleftmargin=1.5mm,
xrightmargin=1.5mm,                        % Caption-position = bottom
breaklines=true,    
escapeinside={@}{@},
                    % Automatic line breaking?
breakatwhitespace=false,                % Automatic breaks only at whitespace?
showspaces=false,                       % Dont make spaces visible
showtabs=false,                         % Dont make tabls visible
columns=flexible,                       % Column format
morekeywords={}
mathescape=false
} 


% Used for displaying a sample figure. If possible, figure files should
% be included in EPS format.
%
% If you use the hyperref package, please uncomment the following line
% to display URLs in blue roman font according to Springer's eBook style:
% \renewcommand\UrlFont{\color{blue}\rmfamily}

\newcommand\parhead[1]{\vspace{1mm}\noindent\textbf{{#1}}\ \ }  
\newcommand\parheadX[1]{\vspace{-1.0mm}\noindent\textbf{{#1}}\ \ }  

\newcommand{\toolurl}[1]{\footnote{\smaller\url{{#1}}\normalsize}}
\newcommand{\ic}[1]{\changefont{cmtt}{m}{n}{#1}\normalfont}  % inline code
\newcommand{\changefont}[3]{\fontfamily{#1}\fontseries{#2}\fontshape{#3}\selectfont}
\newcommand{\fig}[1]{Fig. \ref{#1}}  % inline code
\newcommand{\sect}[1]{Sec. \ref{#1}}  % inline code


%\newcommand\todo[1]{\vspace{1mm}\noindent\textbf{\color{red} {{TODO: {#1}} }}}
\newcommand\todo[1]{}  


\newmdenv[innerlinewidth=1pt, linecolor=black, innerleftmargin=3pt,
innerrightmargin=3pt,innertopmargin=3pt,innerbottommargin=3pt,backgroundcolor=lightlightgray]{mybox}

\newif\ifFullVersion
\FullVersionfalse

\begin{document}
%
\title{Programming \\ vs. \\ That Thing Subject Matter Experts Do}

\titlerunning{Programming vs. SME'ing}

%  \titlerunning{Abbreviated paper title} If the paper title is too long for the
% running head, you can set an abbreviated paper title here
\author{Markus Voelter}
\authorrunning{M. Voelter}
% First names are abbreviated in the running head.
% If there are more than two authors, 'et al.' is used.
\institute{independent/itemis, Oetztaler Strasse 38, 70327 Stuttgart, Germany\\
\email{voelter@acm.org}}
\maketitle              % typeset the header of the contribution
\begin{abstract}
Allowing subject matter experts to directly contribute their domain knowledge
and expertise to software through DSLs and automation is a promising way to increase
overall software development efficiency and the quality of the product. However,
there are doubts of whether this will force subject matter experts to become
programmers. In this paper I answer this question with ``no''. But at the same
time, subject matter experts have to learn how to communicate clearly and 
unambiguously to a computer, and this requires \emph{some} aspects of what
is traditionally called programming. The main part of this paper discusses
what these aspects are and why learning these does not make people programmers. 
\keywords{Domain Specific Language  \and End-user Programming \and Language Engineering.}
\end{abstract}
%
%
%

% \begin{mdframed}
% \textbf{A note for the ISOLA reviewers.} In my experience, one of the biggest challenges
% of the DSL community is that we have a very hard time convincing people who
% might benefit from using DSLs to actually use one.
% We have all written lots of scientific papers and case studies that demonstrate
% clearly that DSLs are useful, but still we have a hard time breaking through. With
% this paper I'm trying a different approach. It is based on anecdotes, it is hopefully
% easy to read, and it doesn't contain any technical details or data.
% I know that this is not a very
% scientific approach. But I would ask you to read the paper and provide feedback
% relative to the goal: to try and convince programmers to give DSLs a try.
% Because---again---I think this is really what our community needs to move forward.
% \end{mdframed}

\newcommand{\datev}{DATEV}
\newcommand{\voluntis}{Voluntis}
\newcommand{\ohb}{OHB}

\newcommand{\caseend}{$\blacksquare$}

 


\section{The role of subject matter experts}

Subject matter experts, or SMEs, own the knowledge and expertise that is the
backbone of software and the foundation of digitalization. But too often this
rich expertise is not captured in a structured way and gets lost when
translating it for software developers (SDs) when they implement it. With the
rate of change increasing, time-to-market shortening and product variability
blooming, this indirect approach of putting knowledge into software is
increasingly untenable: it causes delays, quality problems and frustration for
everyone involved. A better approach is to \emph{empower} subject matter experts
to capture, understand, and reason about data, structures, rules, behaviors and
other forms of knowledge and expertise in a precise and unambiguous form by
providing them with \emph{tailored software languages} (DSLs) and tools that
allow them to directly edit, validate, simulate and test that knowledge. The
models created this way are then automatically transformed into program code.
The software engineers focus their activities on building these
languages, tools and transformations, plus robust execution platforms for the
generated code. \fig{bigpic} shows the overall process.

\section{Can SMEs use DSLs?}

This paper is not about justifying the approach from a technical or economical
perspective. Instead I want to focus on whether the SMEs are \emph{able} to
change from their typically imprecise, non-formal approrach of specifying
requirements using Word, Excel, User Stories or IBM Doors to this DSL-based
approach.

Based on my experience over the years~\cite{infoq} my
answer to this question is a clear yes, at least for the majority
of subject matter experts I have worked with. But a key question is: to what
extent do the subject matter experts who use DSLs have to become
programmers? Do they have the \emph{skills} to be programmers (hint: most
don't), and do they \emph{want} to become programmers (hint: most don't). But we
still expect them to use ``languages'' and IDE-like tools. So:

\vspace{2mm}
\noindent \textbf{Which parts of programming do they have to learn?
How is SME'ing different from programming, and where does it overlap?}
\vspace{2mm}

\noindent I answer this question in \sect{canTheyDoIt}. To set the state we
briefly discuss the domains in which the approach works (\sect{where}) and how
languages and applications are typically architected in such scenarios
(\sect{langarch}).
We conclude the paper with a wrap up in \sect{wrapUp}.

\begin{figure}
\begin{center}
    \includegraphics[width=1\columnwidth]{figures/bigpic.png}
    \caption{The process from the SME's brain into software, based on
    tools and platforms developed by software engineers.}
    \label{bigpic}
\end{center} 
\end{figure} 

\todo{Refer to Manifesto when it is done}

\section{(Where) Does this work?}
\label{where}

Building the necessary tooling and automation requires investment, and this
investment must pay off to make economical sense. Which is why this
approach only works in domains that have the following characteristics. First,
the subject matter has to have a minimum
\textbf{size} and \textbf{complicatedness}. Often this means that there are
people in the company who consider themselves \textbf{experts} in that
subject matter. It is they who everybody asks about details in the domain.
The second criterion is that this subject matter as a whole will remain
\textbf{relevant over time} and that the business intends to continue developing
software in that domain for a reasonably long time. Finally, even though the
subject matter as a whole is relevant for a long time, a degree of
\textbf{evolution} or \textbf{variety} within the domain is usually needed for
the approach to make sense. I have seen this approach used in the following
domains, among others:

\parhead{In insurances,} DSLs are used by insurance product definition staff to
develop a variety of continuously evolving insurance products~\cite{zurich}.
With increased differentiation and tailoring of products, these become more and
more complicated while at the same time increasing in variety and number. The
company itself is in the insurance business for the long run. 

\todo{Reference to
detailed case study coming once MPS book is published}

\parhead{In healthcare,} DSLs are used by doctors and other healthcare
professionals to develop treatment and diagnostics algorithms that for the core
of digital therapeutics apps~\cite{voelter2019vol}. My customer, Voluntis, 
intends to grow over years
and develop a large number of these algorithms and apps. The subject matter is
large and complicated because it captures medical expertise. 

\parhead{In public adminstration,} government agencies are certainly in it for
the long run, while legislation for public benefits and tax calculation changes
and evolves regularly. The agencies are full of experts who use DSLs~\cite{dta} 
to disambiguate and
formalize the law and its interpretations by
courts. Similarly, the service providers who
develop software for tax advisors has the same challenges and uses DSLs as well~\cite{datevsteuervideo,datevsteuerslides}.


\parhead{In payroll,} the regulations that govern the deducations from an
employee's gross salary and the additional taxes and fees they have to pay are
just as complicated, long-lasting and ever changing as tax law (and of course
directly related). Service provides who develop payroll software therefore also
employ whole departments full of experts and benefit from the use of DSLs.
\todo{Reference to detailed case study coming once MPS book is published}

\vspace{3mm}

\noindent For an overview over this approach and a couple of easily readable 
case studies, see my InfoQ article~\cite{infoq}.


\section{Typical DSL Architecture}
\label{langarch}


Many of the DSL I have built follow the general approach that is outlined in
\fig{arch}. The models created by the SMEs end up as the core of the system,
usually expressed with functional semantics, either via generation or via
interpretation. That core implements a (manually defined) API that is used by a
driver component to invoke the DSL-derived code. The driver interacts with
users and other systems -- potentially via additional architectural building
blocks -- and is often also responsible for managing and persisting state.
Indeed, many of our DSLs are funclarative~\cite{voelter2018fusing} where small,
simple calculations are expressed with a functional-style language (so users don't
have to care about effects at this level), and when things get more complicated,
the DSL provides declarative first-class concepts to express those concisely
without lots of low-level functional code.

In order to avoid reinventing the wheel with regards to the core functional
expressions, the DSLs often embed (and then extend) a reusable language
KernelF~\cite{volter2018design}.

If the DSL cannot be scoped to handling only the functional parts and thus has
to manage state, I usually rely on variations of state machines. Generally it is
a good idea to rely on established programming paradigms and DSL-ify them
instead of trying to invent new fundamental paradigms.



\begin{figure}[t]
\begin{center}
    \includegraphics[width=1\columnwidth]{figures/arch.png}
    \caption{Typical high-level systems architecture, where the models
    expressed with DSLs are transformed into code that forms the core
    of a larger application.}
    \label{arch}
\end{center} 
\end{figure} 



\section{Difference between programming and SME'ing}
\label{canTheyDoIt}




\subsection{Skills and Responsibilities}

\begin{figure}[t]
\begin{center}
    \includegraphics[width=1\columnwidth]{figures/table-respo.png}
    \caption{Comparing programming (what software engineers do) from
    whatever not-yet-named activity subject matter experts should do to 
    directly contribute to software. Darker shade means ``is more relevant''.}
    \label{table-respo}
\end{center} 
\end{figure} 

\fig{table-respo} shows the differences in responsibilities of SDs and SMEs. The
darker the shade, the more responsibility the respective community has for that
concern. Let's start our discussion with the two black and white cases, those
where there is no overlap.


\parhead{Region 1, SME only} We begin with region 1, which is completely the
responsibility of SMEs. They have to understand every particular example, case,
situation, and exception of the subject matter they want to describe with the DSL.
This is their natural responsibility, this is why they exist. SDs on the other
hand should not have to care at all. Achieving this separation -- and then
optimizing the tasks of both communities -- is the reason for using DSLs and
tools in the first place. Related to this, the SMEs are also in charge of
determining what consistutes correct behavior in the subject matter:
SMEs take responsibility for what should go into test cases as well as for their
completeness.

\parhead{Region 2, SE only} Let's move on to region 2, which is completely the
responsibility of SDs.
Building and operating automated CI pipelines that build and package the
software and run tests is nothing the SME should be concerned with, except maybe
getting emails if tests fail (after they have run correctly in their local
environment, otherwise they should never reach the CI server).
 
The same is true for taking care of performance, scalability, 
safety, security, robustness and availability, all the operational
(aka non-functional) concerns of the final software system. Keeping the
subject matter segregated from these technical aspects of software is
a key benefit of the approach, and it is clear that this should be 
handled in platforms, frameworks and code generators -- all clearly
the domain of SDs.

Finally, the development of the DSLs and tools that will then be used by the
SMEs for capturing, analysing and experimenting with subject matter is the
responsibility of SDs. It might not be the domain of \emph{all} of the software
engineers, but maybe only of a certain specialization called language engineers
who specialize in developing languages, IDEs, interpreters and generators. Also,
a very few of the subject matter experts -- I sometimes refer to them as gurus
-- have to help survey, understand, analyse and abstract the domain so that the
language engineers can build the languages. But the regular SME, the dozens or
hundreds that many of our customers employ, are not involved with this task.

\parhead{Region 3, Testing} Let's now look at region 3, one that apparently has
full shared responsibility. However, this is a bit misleading. Indeed, both
communities have to understand the purpose of testing, what a test case is,
and appreciate the notion of coverage, i.e., understanding when they have enough
tests to be (reasonably) sure that there are no more (reasonably few) bugs in
the logic. But of course the SMEs care about this \emph{only} for subject matter
as expressed with the DSL, whereas the SDs care about it \emph{only} in the platform,
frameworks, language implementation and generators. So they both have to
understand testing, but there are no artifacts for which they
have shared responsibility.

\parhead{Region 4, shared skills} This is the most interesting part: all the items
in this region are native to programming and software engineering. So SDs care.
But they are also relevant, to different degrees, to SMES when they use DSLs.
Let's explore them in detail.

Of course the SMEs have to understand the conceptual abstractions of the domain,
because otherwise they cannot use the language.
Understanding abstractions is not easy in general, but because in a DSL these
abstractions are closely aligned with the subject matter, the SMEs -- in my
experience -- are able to understand it. Maybe not every little detail (which is
why the box is dark grey and not black) but sufficiently well to use the
language. The SDs have to understand these abstractions as well, especially the
language engieneers. Those who build the execution platform can usually deal
with a black-box view of the generated code.

A note: the fact that there is shared understanding about the core abstractions
of the subject matter in the domain for which the SMEs and the SDs co-create
software is a major reason why DSLs are so useful here. The language definition,
and its conceptual cousins, the core abstractions, are an unambiguous and
clearly-scoped foundation for productive collaboration between the two
communities. So when I write ``xyz is the responsibility of SME/SD'', it doesn't
mean that there is not a \emph{joint overall responsibility} of both communities
together to deliver useful (in terms of subject matter) and robust (in terms of
operational concerns) software.

Back to region 4. No real-world subject-matter focused DSL I have ever seen can
make do without understanding the notion of values, and some notion of functions
(entities that produce new values from inputs). It doesn't matter whether we're
talking specifically about (textual) functions, (Excel-style) decision tables or
(graphical) dataflow diagrams. The good thing is that essentially everybody has
come across these at school or at university, even
though using functions to assemble larger functionality from pieces and the explicit
use of types is new to many. In practice, teaching the use of functions, at least
in limited complexity situations, is feasible. 

Similarly, the understanding of dot expressions to mean \ic{the member X of
entity Y} or \ic{do Z with entity Y} is very hard to avoid, because working with
parts of things or performing activities on things is ubiquitous. For many SMEs,
this is harder to get used to. We've experimented with literally writing
\ic{<member> of <object>} instead of \ic{<object>.<member>} but this results in
less useful IDE support: with the latter syntax, one can easily scope the code
completion menu to members of \ic{object} because users write the object first.
In contrast, with the former syntax, the code completion for the member has to
show all members of all objects in the system because the context \ic{object} is
not yet specified.

In essentially every domain SMEs have to express decisions and calculations.
So another set of constructs that is hard to avoid is arithmetic, logcial and
and comparison operators (together with types like \ic{number} or \ic{Boolean}), 
as well as notion of a conditional,
such as \ic{if \ldots then} or \ic{switch\{case, case, case\}} -- 
independent of their concrete syntax (text, symbolic or graphical). Once
again, most SMEs have come across these operators at school or university, so using
them is not a big challenge. 

The reason why these two lines are grey for SMEs and not totally black is
because the complexity of the expressions built with these language concepts
should (and usually can) be kept lower for SMEs. For example, for complicated
decisions, we can support graphical decision tables of various forms which are
much easier to grasp than nested \ic{if} statements or the equivalent of \ic{switch}
statements. 

Several of the DSLs I have built require an understanding of 
parametrization. For example, in function-like constructs, the values passed
into the function are mapped (by position or by name) to parameters in the
function signature that are then used in the body of the function. Most
SMEs have no problems with this -- again, school experience -- but some do. 
Often parametrization is the threshold where the need for education and training
starts (beyond building the shared understanding about the core concepts of a
domain). A related concept is instantiation, where, usually, each
instance has independent values for its state and its evolution. This is not
taught in school, and it's not taught outside of computer science at university,
so training is needed. On the other hand, many DSLs can do without instantiation
which is why this box is a lighter shade of gray.

We're getting to more advanced concepts that are increasingly harder to grasp
for many SMEs, but they are also not necessary in all DSLs (though unavoidable
in some domains). The notion of specialization or subtyping is key here. While
everybody understands subtyping intuitively (``an eagle \emph{isa} bird
\emph{isa} animal \emph{isa} living thing \emph{isa} object''), many SMEs
struggle with the consequences. Especially the mental assembly of everything
that's in a subtype by (mentally) going through all its supertypes is hard:
``we're not seeing the big picture'' is what I often hear. Practice helps, but
so can tools that optionally show all the inherited members inline in the
subtype's definition.

For complex subject matter -- tax calculation comes to mind -- the models
created with the DSLs become large, and complexity often rises along with size.
Notions like delineating module boundaries, explicitly defined interfaces,
reduction of unnecessary dependencies, and more generally, cohesion and coupling
become an issue. Most SMEs struggle here. But on the other hand, 95\% of the
work of an SME can proceed without caring about these big picture concerns,
except during initial design or downstream review phases, where SD or guru
involvement is often needed so sort these things out. Considering it is only
5\% of the total work, this involvement is usually feasible.

The final ingredient in region 4 is discovering and then defining new
abstractions. This is often not the strong suit of SMEs. Those that are good
at it are usually the gurus who help with language definition, or they have been
assimilated wholly but the software development team. But luckily it is quite
rare that SMEs are required to define new abstractions, because those that are
relevant in the domain should be available first-class in the DSL -- or
retrofitted for the next version once the need becomes obvious.
 


\subsection{Different Emphasis}

In my experience, (most) SMEs prioritize the features of languages and IDEs
different from (most) developers. In this section we'll look at some of the
more prominenet differences, \fig{table-prefs} summarizes them.

\begin{figure}[t]
\begin{center}
    \includegraphics[width=1\columnwidth]{figures/table-prefs.png}
    \caption{SMEs make different trade-offs than software engineers 
    regarding the languages and tools they want to work with.}
    \label{table-prefs}
\end{center} 
\end{figure} 

\parhead{Notation} Developers prefer textual notations, both for their
conciseness, but also for reasons of homogeneity with regards to storing,
editing, diffing and merging code. SMEs, in contrast, tend to emphasize
readability and fit of the notation with established representations in the
domain (e.g., tables in the tax law documents) over these efficiency concerns.
Therefore, if you can build DSLs that are more diverse in notation -- and not
just colored text with curly braces and indentation -- SME buy-in is usually
easier to obtain.

\parhead{Selecting vs. Creating} Developers love the creative freedom of coming
up with an algorithm and crafting their own suitable abstractions from small,
flexible building blocks. SMEs -- including because of their often limited
experience with building their own abstractions -- prefer picking from options
and selecting alternatives.
In my experience SMEs usually accept that they have to read a bit more documentation that
(hopefully!) explains what the different options or alternatives mean.
So DSLs often rely on first-class concepts for the various needs of the domain,
even if this requires the users to first understand what each of them means.
The approach is usually also benefitial for domain-related semantic analysis
(more first class concepts makes it easier to analyze programs) and it is easier
to have a nice notation (because you can associate specific notations with these
first-class concepts). In contrast, programming languages emphasize
orthogonality and composability of their (fewer) first-class concepts.

\parhead{Guidance} A related topic is guidance. Developers are happy with
opening an empty editor and starting to write code. Code completion guides
them a little bit. SMEs prefer more guidance, almost to the point where 
skeleton programs are pre-created after selection from a menu. DSLs that feel
like a mix between a form-based application and program code seem to be 
particularly appreciated by many SMEs. 

\parhead{Tool Support} Taking this further, SDs prefer a toolbox approach, where
the tool offers lots and lots of actions and it is the
developer's job to use each action at the right time, in the right way.
SMEs are more use-case oriented. They want tool support for their typical
workflows and process steps, and specific tool support for each. We have built DSLs that
included wizard-like functionality in the IDE where using the wizard required
more input gestures than just code-completion supported typing. Still the wizard
was preferred by the SMEs.
 
\parhead{Thinking about problems} It is almost a defining feature of programmers
that they think about a problem (and its solution) as a complete algorithm that
can cover all possible execution paths. Sure, tests then validate specific
scenarios, but developers \emph{think} in algorithms. SMEs often think in terms
of examples first, and sometimes exclusively. For example, it is easier for them
to deal with a (hopefully complete) set of sequence diagrams rather than with a
state machine that captures the superset of the sequence diagrams. In terms of
DSL design this means that more emphasis on case distinction in which
separate scenarios are specified separately (even if this incurs a degree of code
duplication) is often a good idea.

\parhead{Validation} Most developers are good at writing tests, writing
them against APIs for a relatively small-size unit, and then running these
tests automatically continuously. SMEs often think of validation more in
terms of ``playing with the system''. They prefer ``simulation GUIs'' over
writing repeatable tests as (a different kind of) program. So build those
simulators first, and then allow the simulator to record ``play sessions''
and persist them as generated test cases for downstream reexecution.

\parhead{Recipe vs. Execution} A program is a recipe which, when combined with
input data, behaves in a particular way. The specific behavior depends on the
input data. So whenever SDs write code, they continuously imagine (and sometimes
try out or trace with the debugger) how the program behaves for (all possible
combinations of) input data. Many SMEs are not very good at doing this. One
reason why Excel is so popular is because it does not make this distinction
between the program and its execution: the program always runs (or,
alternatively, a spreadsheet never runs, it just ``is''). So anything from the
universe of live programming is helpful for DSLs.

\vspace{3mm}
\noindent Despite these differences, there are lots of commonalities as well.
Both communities want good tools (read: IDE support), relevant analyses with
understandable and precise error messages, refactorings and other ways to make
non-local changes to potentially large programs, low turnaround time plus
various ways of illustrating, tracing and debugging the execution of programs.  
However, while software engineers are often willing to compromise on these
features if the expressivity of the language is convincing, SMEs usually will 
not.



\section{Where and how can SMEs learn}

So where and how can SMEs learn the skills from the SME column of \fig{table-respo}? 

\parhead{In school and at university} Ideally everybody should learn these
basics in school and at university. While programming in the strict sense should
be limited to computer science or software engineering curricula, this
``SME'ing'' should be mandatory for everybody, just like reading, writing or
math. Of course such courses shouldn't just teach Java or Python. They should
emphasize the specific skills of ``thinking like a programmer'' with a range of
dedicated and diverse languages and tools.

\parhead{Programming Basics Course} A few years ago, based on the need to
educate and trains a group of SMEs, I created a course called Programming
Basics~\cite{voelterProgBas} that teaches these concepts relevant to SMEs step
by step. It starts with simple values as cells of spreadsheets and then covers
expressions, testing, types, functions, structured values, collections,
decisions and calculations as well as instantiation. The course uses different
varieties and notations for many of these concepts in order to try and emphasize
the concepts. The course is built on MPS and KernelF, and allows extension and
customization on language level towards particular DSLs. We are working on a way
to get this into the browser for easier access.


\parhead{Hedy Language} Felienne Hermans has built Hedy~\cite{hedy}, a gradual
programming language. The goal is to teach ``normal people'' the basics of
programming with a language that grows in capability step by step, with the need
for each next capability motivated by user-understandable limitation in the
previous step. Ultimately, when Hedy is fully developed, it is similar to
Python. Hedy is free and works in the browser.


\parhead{Computational Thinking} In the 2000s, a community of software engineers
came up with the term compuational thinking~\cite{denning2019computational} as
the ``mental skills and practices for designing computations that get computers
to do jobs for people, and explaining and interpreting the world as a complex of
information processes.'' So the idea is similar to what I am advocating,
although the relationship to DSLs and subject matter is missing. Computational
thinking has been critizised as being just another name for computer science; but
my discussions in this paper should make clear that there's a big difference
between computer science and that thing SMEs should do.

\section{Wrap Up}
\label{wrapUp}

It is almost not worth saying because it is obvious: almost all domains,
disciplines, professions and sciences are becoming increasingly computational.
And market forces require companies -- especially those in the traditional
industrial countries -- to become more efficient. I am confident that providing
``CAD programs for knowledge workers'', i.e., DSL, tools and automation, is an
important building block for future economic success.

With the comparison of programming and ``that thing SMEs should do'' in this
paper I hope to make clear that everybody \emph{doesn't} have to become a
programmer.
But: everybody has to be empowered to communicate the subject matter of their
domain precisely to a computer (using DSLs or other suitable tools). And
therefore, everybody has to learn to think like a programmer at least a little
bit, enough to be able to understand and work with the things in the SME column
of \fig{table-respo}. And we software engineers have to adopt this
subject-matter centric mindset and develop languages and tools that are built in
line with the SME preferences in \fig{table-prefs}.

\section*{Acknowledgements}

Thanks to Yulia Komarov and Federico Tomassetti for providing feedback on 
an earlier version of this paper.




\bibliographystyle{abbrv}
\bibliography{doc}

\end{document}